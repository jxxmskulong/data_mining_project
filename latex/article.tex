%%%%%%%%%%%%%%%%%%%%%%%%%%%%%%%%%%%%%%%%%
% Arsclassica Article
% LaTeX Template
% Version 1.1 (10/6/14)
%
% This template has been downloaded from:
% http://www.LaTeXTemplates.com
%
% Original author:
% Lorenzo Pantieri (http://www.lorenzopantieri.net) with extensive modifications by:
% Vel (vel@latextemplates.com)
%
% License:
% CC BY-NC-SA 3.0 (http://creativecommons.org/licenses/by-nc-sa/3.0/)
%
%%%%%%%%%%%%%%%%%%%%%%%%%%%%%%%%%%%%%%%%%

%----------------------------------------------------------------------------------------
%	PACKAGES AND OTHER DOCUMENT CONFIGURATIONS
%----------------------------------------------------------------------------------------

\documentclass[
	12pt, % Main document font size
	a4paper, % Paper type, use 'letterpaper' for US Letter paper
	oneside, % One page layout (no page indentation)
	%twoside, % Two page layout (page indentation for binding and different headers)
	headinclude,footinclude, % Extra spacing for the header and footer
	BCOR5mm, % Binding correction
]{scrartcl}

%%%%%%%%%%%%%%%%%%%%%%%%%%%%%%%%%%%%%%%%%
% Arsclassica Article
% Structure Specification File
%
% This file has been downloaded from:
% http://www.LaTeXTemplates.com
%
% Original author:
% Lorenzo Pantieri (http://www.lorenzopantieri.net) with extensive modifications by:
% Vel (vel@latextemplates.com)
%
% License:
% CC BY-NC-SA 3.0 (http://creativecommons.org/licenses/by-nc-sa/3.0/)
%
%%%%%%%%%%%%%%%%%%%%%%%%%%%%%%%%%%%%%%%%%

%----------------------------------------------------------------------------------------
%	REQUIRED PACKAGES
%----------------------------------------------------------------------------------------

\usepackage[
nochapters, % Turn off chapters since this is an article        
beramono, % Use the Bera Mono font for monospaced text (\texttt)
eulermath,% Use the Euler font for mathematics
pdfspacing, % Makes use of pdftex’ letter spacing capabilities via the microtype package
dottedtoc % Dotted lines leading to the page numbers in the table of contents
]{classicthesis} % The layout is based on the Classic Thesis style

\usepackage{arsclassica} % Modifies the Classic Thesis package

\usepackage[T1]{fontenc} % Use 8-bit encoding that has 256 glyphs

\usepackage[utf8]{inputenc} % Required for including letters with accents

\usepackage{graphicx} % Required for including images
\graphicspath{{Figures/}} % Set the default folder for images

\usepackage{enumitem} % Required for manipulating the whitespace between and within lists

\usepackage{lipsum} % Used for inserting dummy 'Lorem ipsum' text into the template

\usepackage{subfig} % Required for creating figures with multiple parts (subfigures)

\usepackage{amsmath,amssymb,amsthm} % For including math equations, theorems, symbols, etc

\usepackage{varioref} % More descriptive referencing

%----------------------------------------------------------------------------------------
%	THEOREM STYLES
%---------------------------------------------------------------------------------------

\theoremstyle{definition} % Define theorem styles here based on the definition style (used for definitions and examples)
\newtheorem{definition}{Definition}

\theoremstyle{plain} % Define theorem styles here based on the plain style (used for theorems, lemmas, propositions)
\newtheorem{theorem}{Theorem}

\theoremstyle{remark} % Define theorem styles here based on the remark style (used for remarks and notes)

%----------------------------------------------------------------------------------------
%	HYPERLINKS
%---------------------------------------------------------------------------------------

\hypersetup{
%draft, % Uncomment to remove all links (useful for printing in black and white)
colorlinks=true, breaklinks=true, bookmarks=true,bookmarksnumbered,
urlcolor=webbrown, linkcolor=RoyalBlue, citecolor=webgreen, % Link colors
pdftitle={}, % PDF title
pdfauthor={\textcopyright}, % PDF Author
pdfsubject={}, % PDF Subject
pdfkeywords={}, % PDF Keywords
pdfcreator={pdfLaTeX}, % PDF Creator
pdfproducer={LaTeX with hyperref and ClassicThesis} % PDF producer
} % Include the structure.tex file which specified the document structure and layout

\hyphenation{Fortran hy-phen-ation} % Specify custom hyphenation points in words with dashes where you would like hyphenation to occur, or alternatively, don't put any dashes in a word to stop hyphenation altogether

%----------------------------------------------------------------------------------------
%	TITLE AND AUTHOR(S)
%----------------------------------------------------------------------------------------

\title{\normalfont\spacedallcaps{Generazione delle categorie di wikipedia attraverso il clustering}} % The article title

\author{\spacedlowsmallcaps{Cazzaro Dalla Cia Lovisotto Vianello}}

\date{} % An optional date to appear under the author(s)

%----------------------------------------------------------------------------------------

\begin{document}

%----------------------------------------------------------------------------------------
%	HEADERS
%----------------------------------------------------------------------------------------

\renewcommand{\sectionmark}[1]{\markright{\spacedlowsmallcaps{#1}}} % The header for all pages (oneside) or for even pages (twoside)
%\renewcommand{\subsectionmark}[1]{\markright{\thesubsection~#1}} % Uncomment when using the twoside option - this modifies the header on odd pages
\lehead{\mbox{\llap{\small\thepage\kern1em\color{halfgray} \vline}\color{halfgray}\hspace{0.5em}\rightmark\hfil}} % The header style

\pagestyle{scrheadings} % Enable the headers specified in this block

%----------------------------------------------------------------------------------------
%	TABLE OF CONTENTS & LISTS OF FIGURES AND TABLES
%----------------------------------------------------------------------------------------

\maketitle % Print the title/author/date block

\newpage

%----------------------------------------------------------------------------------------
%	INTRODUCTION
%----------------------------------------------------------------------------------------

\section{Introduzione}

- descrizione del problema in generale e del constesto

- delineazione degli obiettivi:

	- gli articoli di wikipedia sono clusterizzabili?

	- rapporto tra cluster e le categorie?

	- cosa succede con varie tecniche di clustering?


\section{Dataset e Analisi Preliminare}
	Il dataset che abbiamo utilizzato per effettuare questa analisi non è altro che un \emph{dump} di wikipedia.
	Tale \emph{dump} è composto da un JSON contenente centomila articoli di Wikipedia in lingua inglese. Per ogni articolo di Wikipedia abbiamo a disposizione il titolo, il testo, l'id e le categorie assegnate all'articolo dai suoi autori.

	La nostra analisi mira a valutare la relazione semantica tra gli articoli e le loro categorie, quindi prima di effettuare l'analisi si è deciso di eliminare tutti gli articoli che non risultano essere associati a nessuna categoria. Queste pagine sono dette \emph{di disambiguazione} e quindi non sono utili per i nostri scopi.

	//todo analisi delle categorie (sort e distribuzione delle categorie)

\section{Rappresentazione del Dataset}
	Prima di procedere con qualsiasi operazione sull'intero corpus si è deciso di preprocessare il data set eliminando le cosiddette \emph{stop words} presenti nel testo e lemmatizzando le parole rimaste. Per l'eliminazione delle \emph{stop words} ci siamo basati su una lista di parole fornita dal sito http://www.ranks.nl/stopwords.

	\subsection{VETTORIALIZZIONE DEGLI ARTICOLI}
		%allenamento word2vec con i suoi vari parametri (dare una motivazione per la dimensione 100)
		Per interagire con i più comuni algoritmi di clustering, ad esempio K-means, si è reso necessario trasformare gli articoli di Wikipedia in vettori.
		Per fare ciò abbiamo adottato una tecnica nota in letteratura con il nome Word2Vec. Tale strumento, ideato da Tomas Mikolov, non è altro che una rete neurale a due strati il cui scopo è quello di trasformare parole del linguaggio naturale in vettori. Nello spazio vettoriale generato, le parole semanticamente più simili saranno più vicine, mentre parole che esprimono concetti differenti risulteranno distanti.

		Tale funzionalità è già implementata nella suite software di Spark, ma per sfruttarla al meglio è necessario settare alcuni parametri. Tra questi uno dei più importanti è la dimensione del vettore in uscita.
		In letteratura si è valutato che una dimensionalità nel ordine dei 100 - 300 cite{w2vdim} risulta un buon compromesso tra prestazioni e capacità di descrivere il corpus.

		Dopo aver allenato il modello Word2Vec sul nostro corpus di testi, trasformando così le singole parole in vettori, ad ogni articolo è stata associata la media vettoriale di tutte le parole presenti nel suo testo. Il \emph{Basic Linear Algebra Subprograms} di Spark ci ha permesso di eseguire questa operazione molto rapidamente.

	\subsection{BAG OF WORDS}

		//todo: la trasformazione in bag of words



\section{Clustering}

	\subsection{Tecniche di Clustering}

		\subsubsection{Hopkins Statistic}
			Riportiamo lo score che ci dice che il nostro dataset è ben clusterizzabile

		\subsubsection{Kmeans}
			Kmeans e il suo score con un bel grafico

		\subsubsection{Altri metodi}

			Abbiamo provato anche il clustering gerarchico e il Gaussian Mixture Model
			ma non abbiamo abbastanza potenza di calcolo

		\subsubsection{Latent Dirichlet Allocation}

			Vediamo cosa viene fuori e un bel grafico


	\subsection{Valutazioni del Clustering}

		\subsubsection{Simple Silhouette}
			Utilizzo della versione semplificata di Silhouette con i centroidi
			Rimozione dei cluster con un solo articolo dal punteggio
			Magari buttiamoci un peso a sta metrica

		\subsubsection{Normalized Mutual Information}
			L'informazione mutua è una quantità che misura la mutua dipendenza di due variabili aleatorie, ovvero quanta informazione porta sull'altra la conoscenza del valore di una delle due.

			Questa misura può essere impiegata per valutare quanto due partizioni, o \emph{clustering}, concordano nel suddividere un set di punti \cite{Manning}.

			Per fare questo, ad ogni cluster è stata associata una variabile indicatrice $\omega$, che assume valore 1 se il punto considerato appartiene al cluster e 0 altrimenti.
			Ogni clustering viene perciò individuato dall'insieme $\Omega$ di queste variabili aleatorie mutualmente esclusive e a somma unitaria.

			Con questa descrizione del problema è possibile calcolare l'informazione mutua tra due distinti clustering $\Omega$ e $\Phi$.
			Questa matrica è stata normalizzata in $(0, 1)$ per garantire un confronto alla pari tra clustering di dimensione diversa.

			Quindi l'informazione mutua normalizzata si calcola come
			\begin{equation} \begin{aligned} \label{eq:NMI}
				& NMI(\Omega, \Phi) = \frac
					{I(\Omega, \Phi)}
					{\left[ H(\Omega) + H(\Phi)\right] / 2} \\
				& \text{dove} \\
				& I(\Omega, \Phi) =
					\sum_{\omega \in \Omega} \sum_{\phi \in \Phi}
						P(\omega \cap \phi) \log \frac {P(\omega \cap \phi)} {P(\omega) P(\phi)} \\
				& H(\Omega) = - \sum_{\omega \in \Omega} P(\omega) \log P(\omega) \\
				& H(\Phi) = - \sum_{\phi \in \Phi} P(\phi) \log P(\phi) \\
			\end{aligned} \end{equation}
			dove come valore delle probabilità di appartenenza sono stati impegate le stime a massima verosimiglianza, per esempio
			\begin{equation*}
				P(\omega) = \frac
					{ \text{numero di punti in }\omega }
					{ \text{numero di punti totali} }
			\end{equation*}.

			\bigbreak

			Purtroppo questa definizione non è direttamente applicabile al confronto tra cluster e categorie perché, mentre i clustering ottenuti con KMeans e LDA sono delle effettive partizioni del dataset, non si può dire lo stesso delle categorie, dato che un articolo può possederne più di una.

			% TODO link o descrizione della IDF utilizzata
			Il primo approccio per scogliere questo nodo è stato quello di eseguire un \emph{ranking} con \emph{Inverse Document Frequency} tra le categorie di ciascun articolo per eleggere la più rappresentativa.
			Grazie a questo passaggio NMI viene calcolata come da equazione \ref{eq:NMI}, con i seguenti risultati al variare del numero di cluster.

			% TODO specificare meglio la normalizzazione
			Il secondo approccio tenta invece di estendere NMI al caso di cluster che si sovrappongano l'uno all'altro, considerando quindi ogni classe $c$ con la sua complementare $\bar{c}$ una partizione dell'insieme dei punti.
			L'informazione mutua viene calcolata quindi per ogni clustering $C = {c, \bar{c}}$ e poi sommata per estrarre un punteggio. A causa di questa passaggio si perde la normalizzazione tra 0 e 1: i risultati sono comunque confrontabili, perché risultano tutti multipli di un fattore che è funzione delle sole categoerie, quindi indipendente dalle varie tecniche impiegate.

\section{Risultati}

	\subsection{Numero di cluster}
		Il K selezionato dalle due tecniche Kmeans e LDA
		Speriamo sia simile!

	\subsection{Validazione con Simple Silhouette}

		L'andamento sempre crescente della Silhouette
		Un bel grafico lineare

	\subsection{Confronto tra i Cluster ottenuti con Normalize Mutual Information}
		Confronto tramite NMI delle due tecniche Kmeans e LDA

\section{Conclusioni}
	Le conclusioni generali dalle analisi effettuate
	Proposte di punti da approfondire in studi futuri

%------------------------------------------------------------------------------
%	BIBLIOGRAPHY
%------------------------------------------------------------------------------

\begin{thebibliography}{12}

\bibitem{w2vdim}
    Tomas Mikolov, Kai Chen, Greg Corrado, and Jeffrey Dean. Efficient estimation of word representations
    in vector space. ICLR Workshop, 2013.

\bibitem{Manning}
	Christopher D. Manning, Prabhakar Raghavan and Hinrich Schütze, Introduction to Information Retrieval, Cambridge University Press. 2008

\end{thebibliography}

%----------------------------------------------------------------------------------------

\end{document}
